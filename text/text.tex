\documentclass{article}
\usepackage{a4wide}
\usepackage[utf8]{inputenc}
\usepackage{amsmath}
\usepackage{listings}
\usepackage{color}
 
\definecolor{codegreen}{rgb}{0,0.6,0}
\definecolor{codegray}{rgb}{0.5,0.5,0.5}
\definecolor{codepurple}{rgb}{0.58,0,0.82}
\definecolor{backcolour}{rgb}{0.95,0.95,0.92}
 
\lstdefinestyle{mystyle}{
    backgroundcolor=\color{backcolour},   
    commentstyle=\color{codegreen},
    keywordstyle=\color{magenta},
    numberstyle=\tiny\color{codegray},
    stringstyle=\color{codepurple},
    basicstyle=\footnotesize,
    breakatwhitespace=false,         
    breaklines=true,                 
    captionpos=b,                    
    keepspaces=true,                 
    numbers=left,                    
    numbersep=5pt,                  
    showspaces=false,                
    showstringspaces=false,
    showtabs=false,                  
    tabsize=2
}
 
\lstset{style=mystyle}
 
\begin{document}
    \section{Matematická formulace problému}
    \paragraph{} Mám řešit problém hledání epicentra zemětřesení ze znalosti L-vln
        zjištěných ze seismografu. Neboli znám časy $t_{1}$, $t_{2}$, $t_{3}$ ve kterých
        naměřili jednotlivé stanice signál a polohy $x_1 = (\theta_{1}, \phi_{1})$, 
        $x_2 = (\theta_{2}, \phi_{2})$, $x_3 = (\theta_{3}, \phi_{3})$ těchto stanic. Vzdálenost
        dvou bodů na sféře se určí jako

    \begin{equation}
        d(x_{i} = (\theta_{i}, \phi_{i}), x_{j} = (\theta_{j}, \phi_{j})) = R \arccos \left[ \cos \theta_{i} \cos \theta_{j} \cos(\phi_{i} - \phi_{j}) + \sin \theta_{i} \sin \theta_{j} \right]
    \end{equation}
        
    \paragraph{} Označím si polohu epicentra jako $s = (\theta_{s}, \phi_{s})$. Vzhledem k tomu, že
        předpokládáme stejnou rychlost šíření vlny ve všech směrech, platí mezí vzdálenostmi 
        stanic a epicentra.

    \begin{equation}
        d(x_1, s) = v t_1
    \end{equation}
    \begin{equation}
        d(x_2, s) = v t_2
    \end{equation}
    \begin{equation}
        d(x_3, s) = v t_3
    \end{equation}

    Z rovnic vyloučím $v$ podělením rovnic.

    \begin{equation}
        \frac{d(x_1, s)}{d(x_2, s)} = \frac{t_1}{t_2}
    \end{equation}

    \begin{equation}
        \frac{d(x_1, s)}{d(x_3, s)} = \frac{t_1}{t_3}
    \end{equation}

    Problém, který budu řešit, bude nelineární soustava dvou rovnic o dvou neznámých 
    $s = (\theta_{s}, \phi_{s})$.

    \begin{equation}
        \frac{d(x_1, s)}{d(x_2, s)} - \frac{t_1}{t_2} = 0
    \end{equation}

    \begin{equation}
        \frac{d(x_1, s)}{d(x_3, s)} - \frac{t_1}{t_3} = 0
    \end{equation}

    \section{Popis použité numerické metody}

    \paragraph{} K řešení využiji Newton-Raphsonovu metodu. Obecně hledám-li řešení soustavy
    rovnic

    \begin{equation}
        \begin{split}
        f_1(x_1, ..., x_n) = 0 \\
        ... \\
        f_n(x_1, ..., x_n) = 0 \\
        \end{split}
    \end{equation}

    nebo úspornějí, pokud $\mathbf{x} = (x_1, ..., x_n)^{T}$ a $\mathbf{f}$ je vektor
    funkcí $f_i$

    \begin{equation}
        \mathbf{f}(\mathbf{x}) = \mathbf{0}
    \end{equation}

    pak pro zpřesnění odhadu $\mathbf{x}_{n}$ platí

    \begin{equation}
        \mathbf{x}_{n + 1} = \mathbf{x}_{n} - \mathbf{J}_{f}^{-1}(\mathbf{x}_{n}) 
        \mathbf{f}(\mathbf{x}_{n})
    \end{equation}

    Přitom $\mathbf{J}_{f}$ je jakobián funkcí $\mathbf{f}$ a dá se určit jako

    \begin{equation}
        \mathbf{J}_{f} = 
            \begin{pmatrix}
                \frac{\partial f_1}{\partial x_1} & \cdots & \frac{\partial f_1}{\partial x_n} \\
                \vdots  & \ddots & \vdots  \\
                \frac{\partial f_n}{\partial x_1} & \cdots & \frac{\partial f_n}{\partial x_n} 
            \end{pmatrix}
    \end{equation}

    \section{Zdrojový kód}
    \paragraph{} Nejdříve si naimportuji potřebné symboly a nadefinuji konstantu pro poloměr zeměkoule.

    \begin{lstlisting}[language=Python]
from numpy import arctan2, arccos, arcsin, cos, sin, matrix, subtract, deg2rad, rad2deg, multiply, sqrt
EARTH_RADIUS = 6371000\end{lstlisting}

    \paragraph{} Nyní si vytvořím pomocnou třídu na reprezentaci polohy na sféře v daném čase. Na
    ní si zároveň definuji instanční metodu $d$, která jako argument přijímá jiný objekt stejného
    typu a určí vzdálenost mezi těmito místy na sféře.

    \begin{lstlisting}[language=Python]
class EarthPosition(object):
    def __init__(self, latitude, longitude, time=0):
        self.latitude = deg2rad(latitude)
        self.longitude = deg2rad(longitude)
        self.time = time

    def d(a, b):
        return EARTH_RADIUS * arccos(
            cos(a.latitude) * cos(b.latitude) * cos(a.longitude - b.longitude) + sin(a.latitude) * sin(b.latitude)
        )

    def __repr__(self):
        return "[{0}, {1}]".format(rad2deg(self.latitude), rad2deg(self.longitude))\end{lstlisting}

    \paragraph{} Dále definuje pomocnou funkci pro numerickou derivaci funkce.

    \begin{lstlisting}[language=Python]
def derivative(fun, x):
    h = 1e-6
    return (fun(x + h) - fun(x - h)) / (2 * h)\end{lstlisting}

    \paragraph{} Nakonec samotná třída, která hledá polohu epicentru. V konstruktoru je
        potřeba ji předat trojici objektů typu \textit{EarthPosition}. Funkce
        \textit{test\_function} představuje každou ze dvou funkcí z rovnice
        $f_i(x_1, x_2, \mathbf{x}) = 0$ (obě rovnice mají stejný
        tvar, až na vstupní parametry $x_i$ a $t_i$). Funkce \textit{longitude\_function} a 
        \textit{latitude\_function} jsou pomocné funkce vyššího řádu, které generují funkce
        jednoho argumentu - jedné ze zeměpisných souřadnic. Zadefinoval jsem je, abych mohl
        jednoduše použít funkci pro numerickou derivaci funkce jedné proměnné.

    \paragraph{} Metoda \textit{inverse\_jacoby} počítá inverzní Jacobián, který se použije
        v metodě \textit{solve}. Tato metoda provádí samotné iterace Newtonovy metody, v každé
        iteraci se určí jacobián a vektor funkcí \textit{f\_matrix}, skalárně se vynásobí a z tohoto
        vektoru se určí nová oprava pozice.

    \begin{lstlisting}[language=Python]
class EpicenterSolver(object):
    def __init__(self, x1, x2, x3):
        self.x1 = x1
        self.x2 = x2
        self.x3 = x3

    @classmethod
    def test_function(cls, x1, x2, x):
        return x1.d(x) / x2.d(x) - x1.time / x2.time

    @classmethod
    def longitude_function(cls, x1, x2, latitude):
        return lambda x: cls.test_function(x1, x2, EarthPosition(latitude, x))

    @classmethod
    def latitude_function(cls, x1, x2, longitude):
        return lambda x: cls.test_function(x1, x2, EarthPosition(x, longitude))

    def inverse_jacoby(self, x):
        f1_lat = self.latitude_function(self.x1, self.x2, x.longitude)
        f1_lon = self.longitude_function(self.x1, self.x2, x.latitude)

        f2_lat = self.latitude_function(self.x2, self.x3, x.longitude)
        f2_lon = self.longitude_function(self.x2, self.x3, x.latitude)

        a = derivative(f1_lat, x.latitude)
        b = derivative(f1_lon, x.longitude)
        c = derivative(f2_lat, x.latitude)
        d = derivative(f2_lon, x.longitude)

        return multiply(1 / (a * d - b * c), matrix([[d, -b], [-c, a]]))

    def solve(self, s, n):
        """Solve the system of equation by Newton-Raphson method."""

        def iteration(x):
            inverse_jacobi = self.inverse_jacoby(x)
            f_matrix = matrix([
                [self.test_function(self.x1, self.x2, x)],
                [self.test_function(self.x2, self.x3, x)],
            ])
            product = inverse_jacobi.dot(f_matrix)
            return EarthPosition(x.latitude - product.item(0), x.longitude - product.item(1))

        for _ in range(n):
            s = iteration(s)

        return s\end{lstlisting}

    \paragraph{} Nakonec počítám epicentrum pro hodnoty odečtené z grafu - toto by šlo samozřejmě
        napsat obecně a vstup přijímat např. jako CLI argumenty nebo vstup ze souboru, pro 
        moje účelý to ale bylo zbytečné. Počáteční odhad jsem zvolil naivně jako průměr zadaných
        zeměpisných souřadnic.

    \begin{lstlisting}[language=Python]
def starting_estimate(x1, x2, x3):
    return EarthPosition(
        (x1.latitude + x2.latitude + x3.latitude) / 3,
        (x1.longitude + x2.longitude + x3.longitude) / 3,
    )


if __name__ == "__main__":
    x1 = EarthPosition(61.601944, -149.117222, 7.5) # Palmer, Alaska
    x2 = EarthPosition(39.746944, -105.210833, 23) # Golden, Colorado
    x3 = EarthPosition(4.711111, -74.072222, 44) # Bogota, Columbia

    solver = EpicenterSolver(x1, x2, x3)
    solution = solver.solve(starting_estimate(x1, x2, x3), 1000)
    print(solution)\end{lstlisting}

    \section{Výsledky}
    \paragraph{} Po spuštění dostanu výsledek.

    \begin{lstlisting}[language=Bash]
bash\$ python app.py
[51.8347986915, 54.5086392079]\end{lstlisting}

    \paragraph{} Po porovnání vzdáleností s grafem dostávám značné odchylky ve vzdálenostech
    epicentra a stanic. Provedl jsem ještě jeden test přidáním testovací funkce
    na zjištění konvergence při malých odchylkách souřadnic a podle výsledků se metoda v 
    pořádku odchyluje pouze a malé hodnoty souřadnic. Provedl jsem i testování s jinou
    funkcí vzdálenosti na sféře, zde ale problém také není.

    \begin{lstlisting}[language=Python]
def stability_testing(last_solution):
    for x in range(0, 10):
        for y in range(0, 10):
            x1 = EarthPosition(61.601944 + x * 0.1, -149.117222 + y * 0.1, 7.5) # Palmer, Alaska
            x2 = EarthPosition(39.746944, -105.210833, 23) # Golden, Colorado
            x3 = EarthPosition(4.711111, -74.072222, 44) # Bogota, Columbia
            solver = EpicenterSolver(x1, x2, x3)
            solution = solver.solve(starting_estimate(x1, x2, x3), 1000)
            print(EarthPosition(
                solution.latitude - last_solution.latitude,
                solution.longitude - last_solution.longitude,
                0
            ))\end{lstlisting}

    \paragraph{} Nakonec jsem při testování zjistil, že i malé odchylky pro 
    zeměpisné souřadnice zadaných stanice vyvolají rozdíli i tisíce mil pro výsledné vzdálenosti 
    stanic a epicentra - získané nepřesnosti jsou tedy pravděpodobny způsobeny nepřesnostmi 
    v zadaných polohách a časech (kvůli velkému poloměru země).
\end{document}
